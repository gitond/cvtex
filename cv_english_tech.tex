\documentclass{article}

% Libraries
\usepackage[a4paper,margin=1in]{geometry} % Set page size and margins
\usepackage{ragged2e} % For alignment
\usepackage{graphicx} % For any future images or enhancements
\usepackage{hyperref} % For clickable links
\usepackage{tabularx} % For table formatting

\begin{document}
\noindent
\textbf{\LARGE Botond Ortutay} \hfill \textbf{\LARGE CURRICULUM VITAE}
\vspace{0.5cm} % Space after the header before content

\noindent \hfill March 28, 2025

\vspace{0.3cm} % Small gap between date and text

\noindent
YO-kylä 5B 11 \newline
20540 Turku \newline
Telephone: +358 44 9340057 \newline
boti.ortutay@gmail.com \newline
LinkedIn: \href{https://www.linkedin.com/in/botond-ortutay/}{https://www.linkedin.com/in/botond-ortutay/} \newline
GitHub: \href{https://github.com/gitond/}{https://github.com/gitond/} \newline

\vspace{1cm}
\textbf{\LARGE INTRODUCTION}
\vspace{0.5cm}

\noindent I am currently a student at the University of Turku, although I'll 
finish my Master's Degree soon. In the mean time I've completed a deep dive 
into AI, algorithmics, software architecture and all that good stuff. First 
and foremost I'm a problem solver. If something interests me I find serious 
joy in building solutions.

\vspace{1cm}
\textbf{\LARGE EDUCATION}
\vspace{0.5cm}

\noindent 2024 to present \hspace{1cm}\textbf{Master of Science}, University of Turku, Finland \newline
\hspace*{3.35cm} Major: Software Engineering \newline
\hspace*{3.35cm} Minor: Tomorrow's AI \newline
\vspace{0.5cm}

\noindent 2020-2024 \hspace{1.85cm}\textbf{Bachelor of Science}, University of Turku, Finland \newline
\hspace*{3.35cm} Major: Information and Communication Technology \newline
\hspace*{3.35cm} Minor: Mathematics, Optimization \newline
\vspace{0.5cm}

\noindent 2016-2019 \hspace{1.85cm}\textbf{Tampereen Teknillinen Lukio}, (Tampere High School of Technology) \newline
\hspace*{3.35cm} Mathematical and IT-focused Matek-line \newline

\vspace{1cm}
\textbf{\LARGE PROJECT HIGHLIGHTS}
\vspace{0.5cm}

\textbf{\large graph-tests-2}
\vspace{0.5cm}

\noindent A part of my BSc thesis. Contains implementations of pathfinding 
algorithms in C++ using templates and the Boost Graph Library as well as the 
measured data and calculations concerned with that.

\vspace{0.5cm}
\textbf{\large Imageminer}
\vspace{0.5cm}

\noindent My team's submission for Boost Turku's Digital Sustainability 
Hackathon 2024. It is a (unfinished) tool developed for mobile with C and the 
QR-code recognition library.

\vspace{0.5cm}
\textbf{\large Purgatory}
\vspace{0.5cm}

\noindent My team's submission for the annual GMTK Game Jam in 2023. It was 
written in the Godot game engine using it's own pythonlike GDscript 
programming language. It is playable at 
\href{https://v5000a.itch.io/purgatory}{https://v5000a.itch.io/purgatory}

\vspace{1cm}
\textbf{\LARGE LANGUAGES}
\vspace{0.5cm}

\noindent\begin{tabularx}{\linewidth}{l X}
Hungarian: & native \\
Finnish: & Scored L (top 5\%) in native level matriculation exam test \\
English: & Scored M (top 40\%) in advanced level matriculation exam test \\
Swedish: & Scored M (top 40\%) in intermediate level matriculation exam test \\
\end{tabularx}

\vspace{1cm}
\textbf{\LARGE PROGRAMMING LANGUAGES}
\vspace{0.5cm}

\textbf{\large Python with libraries}
\vspace{0.5cm}

\noindent Throughout my education and my projects I have used Python for many 
tasks, including data processing, AI/ML development, generic software 
development and even game development. Libraries I'm familiar with include 
\textbf{Numpy}, \textbf{Pandas}, \textbf{PyTorch}, \textbf{Tensorflow} as 
well as parts from the Python Standard Library such as math, os and random.

\vspace{0.5cm}
\textbf{\large C/C++}
\vspace{0.5cm}

\noindent I used C++ for writing the project in my BSc thesis, so I mainly for 
Algorithmics. My thesis project was written using the Object Oriented and 
Generic Programming paradigm. Libraries I'm familiar with include libraries 
from \textbf{Boost}, mainly the \textbf{Boost Graph Library}. I'm also 
familiar with C, and used it in my Embedded Systems Programming course at 
Turku University and also a hackathon project written for Android in C.

\vspace{0.5cm}
\textbf{\large Java}
\vspace{0.5cm}

\noindent Java is my most used language in my university education. I have 
used it to write object oriented applications, as well as server backends and 
multithreaded applications. I have also been interested in \textbf{Android 
Programming} and have written Android Apps in Java using \textbf{Android 
Studio}.

\vspace{0.5cm}
\textbf{\large JavaScript}
\vspace{0.5cm}

\noindent My first experience with JavaScript was on a high school programming 
course where we learned the basics. Afterwards I have gone on to use it in my 
own projects, for example the web based game Roboleon. I’ve also studied web 
based development and built web projects as evident from my Github Repo. My 
frameworks include \textbf{React}, \textbf{Vue 3} and \textbf{Node.js}.

\vspace{1cm}
\textbf{\LARGE DATABASES}
\vspace{0.5cm}

\textbf{\large SQL}
\vspace{0.5cm}

\noindent In university I completed the “Introduction to databases” course in 
Turku University, where I was introduced to SQL. I set up an SQL server for 
myself for exercises and experimentation using MariaDB.

\vspace{0.5cm}
\textbf{\large MongoDB}
\vspace{0.5cm}

\noindent During the University of Turku course "Web and Mobile Programming" I 
set up a full-stack application using MongoDB as a database. Since then I've 
also used it in my personal projects

\vspace{1cm}
\textbf{\LARGE OTHER RELEVANT EXPERIENCE}
\vspace{0.5cm}

\textbf{\large Linux}
\vspace{0.5cm}

\noindent I have been a daily Linux (more specifically Fedora) user for more 
than ten years now. It has been my primary operating system on both laptops 
and desktop computers. Nearly all of my software & game development 
experience comes from Linux.

\vspace{0.5cm}
\textbf{\large Open source software}
\vspace{0.5cm}

\noindent As a Linux user I am quite experienced with working with open 
source software. This includes both such software as Gimp, Inksacpe and 
LibreOffice, and also specifically development oriented open source solutions 
(like Git, bash and VSCode).

\vspace{0.5cm}
\textbf{\large UI/UX design}
\vspace{0.5cm}

\noindent I have done university courses on the topic and used tools such as 
Figma to design interfaces and interactions mainly for websites and mobile 
applications.

\end{document}
